\section{Introduction}
\label{sec:introduction}
% The need for PV, and PS.
Our dependence on technology has increased exponentially in recent decades.
%
We use technology in every aspect of our life, from daily mundane tasks like checking the news to life critical and expensive tasks like flying airplanes.
%
The complexity of hardware and software also increased as a result.
%
As we depend more on technology, there is an increasing need to ensure that those ever more complex technologies work as intended, especially in critical applications.
%
To address this concern, engineers utilize \emph{Program Verification} tools
%
to formally (offering mathematical proofs) check if a program satisfies some constraints specified as first-order logic formulas.
%
Those constraints are encodings of the relationships between the inputs and outputs of a program.
%
Such tools have uncovered many bugs and raised the standard of reliability to a new level that is now required in certain industries.
%
Program Verification techniques in some fields have advanced to the point where instead of verifying that a program satisfies some given correctness specifications,
%
they can synthesize a program which satisfies those specifications by construction.

%Potential of PS as well as challenges and shortcomings
Program Synthesis has the potential to impact the quality of software even more than program verification.
%
Program Synthesis, however, is a much harder problem to solve with a high potential for a tool to get stuck on a wrong part of the infinite search space (TODO: give a simple example).
%
To guide the tool and limit the search space the tools are often also provided with a set of syntactic operators (i.e., a context-free grammar) it can compose together to generate programs in the search space (set of allowed programs).
%
The original set of semantic constraints combined with the new set of syntactic constraints constitute a Syntax-Guided Synthesis (SyGuS) Problem~\cite{sygus:2013}.

\begin{figure}[ht!]
    \centering
    \includegraphics[scale=0.8]{images/sygus-solver.pdf}
    \caption{\begin{small}
            %
            The diagram shows the SyGuS solver at work.
            %
            Using the grammar (1), the enumerator and produces a term (2) and sends it to the SMT solver.
            %
            The solver checks the term against the constraints (3) and if satisfied, produces the program output (5).
            %
            Otherwise, asks the enumerator for the next term (4).
        \end{small}}
\end{figure}

