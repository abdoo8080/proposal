\section{Proposal}
\label{sec:proposal}
%
In this project, our main goal is to equip SyGuS solvers with extra functionality that can generate optimal programs according to some given cost constraints.
%
First, we need to communicate to the solver the cost of each term in the search space.
%
We can do so by adding \emph{weight annotations} to the grammar rules of the original formulation of the SyGuS problem.
%
For the matrix multiplication example, the extension would look like
%
\[
    S ::=
    \begin{pmatrix} I & I \\
        I & I
    \end{pmatrix}^{\red{0}}, \quad
    I ::= V \ | \ I +^{\red{1}} I \ | \ I -^{\red{1}} I \ | \ I *^{\red{16}} I, \quad
V ::= a_{ij}^{\red{0}} \ | \ b_{ij}^{\red{0}}.
\]
%
We then define the \emph{total weight} of the program to be the sum of the individual weights of each rule applied to construct the program. For example, $weight(a_{11} * b_{11} + a_{12} * b_{21}) = 4*0 + 1*1 + 2* 16 = 33$.
%

Adding weight annotations to grammar rules changes the original SyGuS problem into an optimization problem:
%
synthesize a program that satisfies the (1) semantic and (2) syntactic constraints while (3) minimizing the total weight.
%
% Our proposal is to solve this new problem by extending current strategies to account for weights.
%
We explore two strategies:
%
\begin{description}
\item[modify enumerators to generate terms with increasing weights:]
%
current enumerators are designed to enumerate all terms in the language specified by the grammar.
%
They start by enumerating the simplest terms (e.g., terminal symbols) and then iteratively apply production rules on them~\cite{cvc4:2015}.
%
In other words, they enumerate terms in order of increasing the number of applied rules.
%
Our aim here is to modify the enumerators to generate terms in order of increasing weights.
%
This modification will force SyGuS solvers to generate minimal weight solutions that satisfy the provided semantic constraints.
%
This change can be thought of as an extension of the current behavior of enumerators,
%
which can be recovered by simply assigning terminal symbols a weight of 0 and other rules a weight of 1.
%
\item[improve unification to generate terms with decreasing weights:]
%
unification is a process of matching the structure of the desired solution against the grammar rules.
%
If a match succeeds, the SyGuS problem is then divided into smaller sub-problems that are easier to solve~\cite{eusolver:2017}.
%
For matrix multiplication problem, we know that the solution is a $2 \times 2$ matrix. So, instead of enumerating all $2 \times 2$ matrices
%
\begin{equation}
\begin{pmatrix} a_{11} & a_{11} \\ a_{11} & \textcolor{blue}{a_{11}} \end{pmatrix},
\begin{pmatrix} a_{11} & a_{11} \\ a_{11} & \textcolor{blue}{a_{12}} \end{pmatrix},
\begin{pmatrix} a_{11} & a_{11} \\ a_{11} & \textcolor{blue}{a_{21}} \end{pmatrix},
...,
\label{eq:sequence-matrices}
\end{equation}
%
we can ``pattern match" the rule $\begin{pmatrix} I & I \\ I & I \end{pmatrix}$ against $AB$ in ~\ref{eqn:matrix-mul} and try to solve the sub-problems
%
\begin{align*}
    mul(A,B)_{11} &= a_{11} * b_{11} + a_{12} * b_{21} &
    mul(A,B)_{12} &= a_{11} * b_{12} + a_{12} * b_{22} \\
    mul(A,B)_{21} &= a_{21} * b_{11} + a_{22} * b_{21} &
    mul(A,B)_{22} &= a_{21} * b_{12} + a_{22} * b_{22}. 
\end{align*}
%
Solving these sub-problems using enumeration is orders--of--magnitude faster than enumerating \eqref{eq:sequence-matrices}.
%
State of the art solvers utilize this technique to trim the size of large search spaces.
%
However, it is not straight forward to support weights with unification.
%
For example, following a greedy approach by choosing rules with minimal weights for pattern matching may not lead us to a minimal solution.
%
One approach to resolve that might be to ignore weights and find an initial solution to the problem using unification.
%
Then, iteratively use enumeration over the initial solution to reduce the weights of the sub-terms.
%
Another open question is what the termination condition for this iterative process would be like.
%
It might be clear for simple problems, but it gets more difficult with more complex problems.
%
% For example, such a solution does not scale for problems like matrix multiplication.
%
A possible compromise could be to relax the weight minimality constraint to accept a range of weights as the termination condition.
%
% instead, we can use some heuristics to recover termination for these type of problems (e.g., rules with 0 weight like $\begin{pmatrix} I & I \\ I & I \end{pmatrix}^0$).
\end{description}