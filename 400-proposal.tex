\section{Proposal}
\label{sec:proposal}
%TODO: make smooth transition to this sentence
In this project, our main goal is to equip SyGuS solvers with extra functionality that can generate optimal programs according to some given cost constraints.
%
Adding weight annotations to grammar rules changes the original SyGuS problem into an optimization problem:
%
synthesize a program that satisfies the (1) semantic (2) syntactic constraints while minimizing the total (3) weight.
%
The main goal of our proposal is to solve this new problem by extending current techniques to account for weights.
%
We explore two strategies to design
%
\begin{description}
\item[modify enumerators to generate terms with increasing weights:]
%
current enumerators are designed to enumerate all terms in the language specified by the grammar.
%
They start by enumerating the simplest terms (e.g., terminal symbols) and then iteratively apply production rules on them.
%
Our aim here is for the enumerators to generate terms in order of increasing weights.
%
This modification will force SuGus solvers to generate minimal weight solutions 
%
that satisfy the provided semantic constraints.
%
% This modification will force the SyGuS solvers to test ``lighter" terms against the semantic constraints first and ensure they return a solution with minimal weight.
%
This change can be thought of as an extension of the current behavior of enumerators, 
%
which can be recovered by simply assigning terminal symbols a weight of 0 and other rules a weight of 1.
%
\item[improve unification to generate terms with decreasing weights:]
%
unification is a process of pattern matching the structure of the desired solution against the grammar rules.
%
If a match succeeds, the SyGuS problem is then divided into smaller sub-problems that are easier to solve.
%
For matrix multiplication problem, we know that the solution is a $2 \times 2$ matrix. So, instead of enumerating all $2 \times 2$ matrices
%
\begin{equation}
\begin{bmatrix} a_{11} & a_{11} \\ a_{11} & \textcolor{blue}{a_{11}} \end{bmatrix},
\begin{bmatrix} a_{11} & a_{11} \\ a_{11} & \textcolor{blue}{a_{12}} \end{bmatrix},
\begin{bmatrix} a_{11} & a_{11} \\ a_{11} & \textcolor{blue}{a_{21}} \end{bmatrix},
...,
\label{eq:sequence-matrices}
\end{equation}
%
we can ``pattern match" the rule $\begin{bmatrix} I & I \\ I & I \end{bmatrix}$ against $AB$ and try to solve the sub-problems
%
\begin{align*}
    mul(A,B)_{11} &= a_{11} * b_{11} + a_{12} * b_{21} &
    mul(A,B)_{12} &= a_{11} * b_{12} + a_{12} * b_{22} \\
    mul(A,B)_{21} &= a_{21} * b_{11} + a_{22} * b_{21} &
    mul(A,B)_{22} &= a_{21} * b_{12} + a_{22} * b_{22}. 
\end{align*}
%
Solving these sub-problems using enumeration is orders--of--magnitude faster than enumerating \eqref{eq:sequence-matrices}.
%
State of the art solvers utilize this technique to trim the size of large search spaces.
%
However, it not clear to us yet how to support weights with unification. 
%
For example, following greedy approach by choosing minimum rules for pattern matching may not lead us to a minimal solution. 
%
One approach to resolve that might be to ignore weights and find an initial solution to the problem using unification.
%
Then, iteratively use enumeration over the initial solution to reduce the weights of the sub-terms.
%
It might be clear how such iterative process may terminate for simple problems,
%
but it gets more difficult with more complex problems. 
%
For example, such solution does not scale for problems like matrix multiplication.
%
instead, we can use some heuristics to recover termination for these type of problems (e.g., rules with 0 weight like $\begin{bmatrix} I & I \\ I & I \end{bmatrix}^0$).
\end{description}