\section{Broader impact}
\label{sec:impact}
Program Synthesis has a wide variety of applications from synthesizing new hardware circuits, to generating SQL queries, to generating secure protocols.
%
SyGuS utilizes the rich SMT framework and its mature tools to synthesize programs with desired correctness and security guarantees.
%
However, we often care about other, sometimes competing, ``soft" constraints.
%

Take hardware synthesis for example, a hardware architect spends much of their time minimizing cost and power consumption of their hardware while maximizing its performance (i.e., minimizing latency).
%
We can encode those 3 constraints as weights on the grammar rules representing individual components of the hardware.
%
We can then use SyGuS solvers to help us compare different designs that minimize each constraint independently or several combinations of those constraints.
%

Another example is synthesis of SQL queries where the main priority is to optimize their performance.
%
Given a set of inputs and the desired output, Sygus can synthesize multiple queries that compute the same output.  
%
But their performance vary depending on the operators used in these queries, and the structure of the underlying tables. 
%
Usually, the generated SQL queries are compiled to execution plans by Database managment systems.
%
Those execution plans consist of small steps like: table scan, filter, hashjoin, sort, etc.
%
Some of those steps, e.g., scan, are more expensive than others, because of latency of external storage.
%
Synthesizing an optimal query can be formulated as a SyGuS problem, where each operator is assigned a specific cost, and the optimization goal is to reduce the number of overall query cost. 

% This is achieved by reducing the amount of data read from persistent storage, exploiting indices on specific columns, using materialized views, etc.
% DBMS prepares execution plans for each query using several predefined operators like: table scan, filter, hashjoin, sort, etc. We can assign weights to these operators


%
% Persistent storage access is a bottleneck in slow SQL queries that operate over large datasets.
%
% We can sometimes mitigate this issue by reordering some of the operations we apply on the tables.
%
% We can either combine the two tables and then filter out names that do not start with an \texttt{A} or filter the two tables first and then combine them. The latter query is preferred as it reduces the number of reads and writes to disk. We can convey this information to SyGuS solvers bu giving low weights to \texttt{UNION} operations applied after \texttt{SELECT} operations and high weights for the opposite case.
%

% TODO: come up with a security example. Either a protocol, its implementation, or maybe talk about firewalls.
